\documentclass[article]{jss}

\usepackage{amsfonts}       % blackboard math symbols
\usepackage{tikz}
\usepackage{amssymb,amsmath}
\DeclareMathOperator*{\argmin}{arg\,min}
\DeclareMathOperator*{\sign}{sign}
\DeclareMathOperator*{\Lik}{Lik}
\DeclareMathOperator*{\Peaks}{Peaks}
\newcommand{\Cost}{\text{Cost}}
%\DeclareMathOperator*{\Cost}{Cost}
\DeclareMathOperator*{\Diag}{Diag}
\DeclareMathOperator*{\TPR}{TPR}
\DeclareMathOperator*{\Segments}{Segments}
\DeclareMathOperator*{\FPR}{FPR}
\DeclareMathOperator*{\argmax}{arg\,max}
\DeclareMathOperator*{\maximize}{maximize} 
\DeclareMathOperator*{\minimize}{minimize}
\newcommand{\ZZ}{\mathbb Z}
\newcommand{\NN}{\mathbb N}
\newcommand{\RR}{\mathbb R}
%%%%%%%%%%%%%%%%%%%%%%%%%%%%%%
%% declarations for jss.cls %%%%%%%%%%%%%%%%%%%%%%%%%%%%%%%%%%%%%%%%%%
%%%%%%%%%%%%%%%%%%%%%%%%%%%%%%

%% almost as usual
\author{Anuj Khare\\IIT Guwahati \And
        Toby Dylan Hocking\\McGill University \And
        Jelle Goeman \\Leiden University}
\title{Regularization Paths for Interval Regression Models via Coordinate Descent}

%% for pretty printing and a nice hypersummary also set:
\Plainauthor{Anuj Khare, Toby Dylan Hocking, Jelle Goeman} %% comma-separated
\Plaintitle{Regularization Paths for Interval Regression Models via Coordinate Descent} %% without formatting
\Shorttitle{Regularization Paths for Interval Regression Models via Coordinate Descent} %% a short title (if necessary)

%% an abstract and keywords
\Abstract{
  We describe a coordinate descent algorithm for estimating interval 
  regression models with elastic net penalties. The output data we consider
  are of four possible censoring types: left, right, interval, and none. 
  The models we consider are Gaussian, Logistic, TODO. We evaluate the speed
  and accuracy of the coordinate descent method on several real data sets,
  including survival prediction and and penalty learning problems.
}
\Keywords{keywords, comma-separated, not capitalized, \proglang{Java}}
\Plainkeywords{keywords, comma-separated, not capitalized, Java} %% without formatting
%% at least one keyword must be supplied

%% publication information
%% NOTE: Typically, this can be left commented and will be filled out by the technical editor
%% \Volume{50}
%% \Issue{9}
%% \Month{June}
%% \Year{2012}
%% \Submitdate{2012-06-04}
%% \Acceptdate{2012-06-04}

%% The address of (at least) one author should be given
%% in the following format:
\Address{
  Achim Zeileis\\
  Department of Statistics and Mathematics\\
  Faculty of Economics and Statistics\\
  Universit\"at Innsbruck\\
  6020 Innsbruck, Austria\\
  E-mail: \email{Achim.Zeileis@uibk.ac.at}\\
  URL: \url{http://eeecon.uibk.ac.at/~zeileis/}
}
%% It is also possible to add a telephone and fax number
%% before the e-mail in the following format:
%% Telephone: +43/512/507-7103
%% Fax: +43/512/507-2851

%% for those who use Sweave please include the following line (with % symbols):
%% need no \usepackage{Sweave.sty}

%% end of declarations %%%%%%%%%%%%%%%%%%%%%%%%%%%%%%%%%%%%%%%%%%%%%%%


\begin{document}

%% include your article here, just as usual
%% Note that you should use the \pkg{}, \proglang{} and \code{} commands.

%\section[About Java]{About \proglang{Java}}
\section{Introduction}
%% Note: If there is markup in \(sub)section, then it has to be escape as above.

There has been a recent interest in Accelerated Failure Time models
\citep{huang05, cai09, khan13}.

The coordinate descent algorithm has been used for elastic net
regularized generalized linear models \citep{glmnet}, including Cox's
proportional hazards model \citep{coxnet}.

\section{Model}

In interval regression problems we are given $n$ observations. For
each observation $i\in\{1, \dots, n\}$, we have a $p$-dimensional
input feature vector $\mathbf x_i \in \RR^p$. The output data are
lower $\underline y_i\in\{-\infty, \RR\}$ and upper
$\overline y_i\in\{\RR,\infty\}$ limits which verify
$\underline y_i \leq \overline y_i$.

For example, usual linear regression is a special case where
$\underline y_i = \overline y_i\in\RR$ for all observations
$i\in\{1, \dots, n\}$. In this case we say that none of the
observations are censored.

Another example is survival data, where each observation $i$
corresponds to a patient who is administered a treatment in a
study. The model predicts the amount of time that a patient survives
after treatment, and there are two types of observed output data:
\begin{itemize}
\item Patient $i$ died before the end of the study, at time $t_i>0$
  after treatment. This implies a log survival time of
  $\underline y_i= \overline y_i = \log t_i$. There is no censoring
  for this observation.
\item Patient $i$ was alive at the end of the study, and the time
  between treatment and the end of the study is $t_i$. This implies
  the patient's log survival time is somewhere in the interval
  $(\underline y_i, \overline y_i) = (\log t_i, \infty)$. This
  observation is right censored.
\end{itemize}

A final example is penalty learning, where the model predicts a
positive penalty constant that is used to select the number of
change-points in segmentation models \citep{HOCKING-penalties}. The
output data are intervals $(\underline y_i, \overline y_i)$ which are
either finite or infinite, but for these data we never have
$\underline y_i = \overline y_i$. In terms of the survival literature,
we say that all of the observations are either left, right, or
interval censored.

\subsection{The concave maximum likelihood problem}

The predicted output is $\beta + \mathbf w^\intercal \mathbf x_i$,
where $\beta\in\RR$ is the bias/intercept, and $w\in\RR^p$ is a weight
vector.

TODO discuss likelihood in terms of the scale $\sigma>0$ and log-scale
$s=\log \sigma\in\RR$.

\subsection{The convex minimization problem}

The regularized model can be written as the following minimization
problem. 
\begin{equation}
  \label{eq:min_cost}
  \minimize_{
    \mathbf w,
    \beta,
    s
  }
  \Cost_{\alpha,\lambda}(\mathbf w, \beta, s).
\end{equation}
The Cost function has two regularization hyper-parameters,
$\lambda\geq 0$ and $\alpha\in[0,1]$, which must be fixed before
solving for the optimization variables $\mathbf w, \beta, s$. The Cost
function is composed of a smooth convex loss $\mathcal L$ and a convex
penalty $P_\alpha$.
\begin{equation}
  \label{eq:Cost_loss_pen}
  \Cost_{\alpha,\lambda}(\mathbf w, \beta, s) =
  \mathcal L(\mathbf w, \beta, s) +
  \lambda P_\alpha(\mathbf w).
\end{equation}
The penalty
$P_\alpha(\mathbf w)=\alpha||\mathbf w||_1 + (1-\alpha)||\mathbf
w||_2^2/2$ is the elastic net, and the average loss $\mathcal L$ is defined as
\begin{equation}
  \label{eq:Loss}
  \mathcal L(\mathbf w, \beta, s) = 
  \frac 1 n \sum_{i=1}^n \ell(
  \underline y_i, 
  \overline y_i,
  s,
  \beta + \mathbf w^\intercal \mathbf x_i).
\end{equation}
Let $\eta_i = \beta + \mathbf w^\intercal \mathbf x_i$.  The observation-specific loss $\ell$ is defined as
\begin{equation}
  \label{eq:ell}
\ell(
  \underline y, 
  \overline y,
  s,
  \eta
)=
\begin{cases}
  s-\log f\left(\frac{\underline y - \eta}{e^s}\right) &
  \text{ if } \underline y = \overline y\\
  -\log\left[
    F\left(\frac{\overline y-\eta}{e^s}\right)-
    F\left(\frac{\underline y-\eta}{e^s}\right)
  \right] & \text{ otherwise.}
\end{cases}
\end{equation}
The losses for left and right censored observations can be obtained by setting
$\underline y = \mathbf {-Inf}$ and $\overline y = \mathbf {Inf}$ respectively.
The density $f$ and cumulative distribution function $F$ depend on the
choice of the distribution (normal, logistic, etc). TODO table with
precise equations for $f$ and $F$ for the models we have implemented.

\section{Algorithm}

We adapt the gradient computations of \citet{survival-manual}. TODO

\section{Results}

We use the code to learn penalty functions that predict the number of
change-points in segmentation models \citep{HOCKING-penalties,
  PeakSeg}. These problems are relatively small in terms of number of
observations $n$ and features $p$.

TODO: is there a really big (in terms of observations or features)
survival data set we could apply our code to?

\bibliography{refs}

\end{document}
